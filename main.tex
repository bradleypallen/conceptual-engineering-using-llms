\documentclass{article}

\usepackage{arxiv}

%\usepackage[utf8]{inputenc} % allow utf-8 input
\usepackage[T1]{fontenc}    % use 8-bit T1 fonts
\usepackage{hyperref}       % hyperlinks
\usepackage{url}            % simple URL typesetting
\usepackage{booktabs}       % professional-quality tables
\usepackage{amsfonts}       % blackboard math symbols
\usepackage{nicefrac}       % compact symbols for 1/2, etc.
\usepackage{microtype}      % microtypography
\usepackage{lipsum}		    % Can be removed after putting your text content
\usepackage{graphicx}
\usepackage{natbib}
\usepackage{doi}
\usepackage{emoji}
\usepackage{rotating}
\usepackage{csquotes}

\title{Conceptual Engineering Using Large Language Models}

%\date{September 9, 1985}	% Here you can change the date presented in the paper title
%\date{} 					% Or removing it

\author{Bradley P. Allen\\
\texttt{bradley.p.allen@gmail.com} \\
}

% Uncomment to override  the `A preprint' in the header
\renewcommand{\headeright}{Draft}
\renewcommand{\undertitle}{Draft}
\renewcommand{\shorttitle}{\textit{Conceptual Engineering Using Large Language Models}}

%%% Add PDF metadata to help others organize their library
%%% Once the PDF is generated, you can check the metadata with
%%% $ pdfinfo template.pdf
\hypersetup{
pdftitle={Conceptual Engineering Using Large Language Models},
pdfsubject={cs.CL},
pdfauthor={Bradley P. Allen},
pdfkeywords={Conceptual engineering, Language models, Knowledge graphs, Artificial intelligence},
}

\begin{document}
\maketitle

\begin{abstract}
In philosophy, \textit{conceptual engineering} is "the design, implementation, and evaluation of concepts" \cite{chalmers2020conceptual}. There are a wide range of views as to the proper definition of the term \textit{concept} and how conceptual engineering can be performed, but in many instances it is an activity where language processing is a significant element. In recent years, large language models (LLMs) have emerged as a technology that has promises to be of "substantial value in the scientific study of language learning and processing" \cite{mahowald2023dissociating}.  In this paper, we ask the question "can large language models support the process of conceptual engineering?" We provide two examples that suggest that large language models can be used to this end. 
\end{abstract}

% keywords can be removed
\keywords{Conceptual engineering \and Language models \and Knowledge graphs \and Artificial intelligence}

\section{Conceptual engineering as natural language processing}

\section{Ameliorative analysis through prompt engineering}

\subsection{An example using Haslanger's definition of \textit{woman}}

\section{Explication using LLMs to translate between evolved and constructed languages}

\subsection{An example using Carnap's explication of \textit{fish} by \textit{piscis}}

\section{Discussion}

\subsection{The relationship of conceptual engineering, ontology engineering, and knowledge engineering}

\subsection{Ethical risks in the use of LLMs for conceptual engineering}

\section{Acknowledgements}

\nocite{*}

\bibliographystyle{unsrt}
\bibliography{references}

\end{document}
